\section{Algoritmo de seleção}
O algoritmo de Seleção é um algoritmo de ordenação no qual, a cada itineração, o menor elemento da lista é garantido estar na posição correta. Ou seja, na primeira itineração, assegura-se que o menor elemento ficará na primeira posição da lista, na segunda itineração, o segundo elemento, e assim por diante. 
\vspace{\baselineskip}

Acerca do seu desempenho, o algoritmo em questão possui um "desempenho" quadrático, isto é, para $n$ elementops da lista, o algoritmo realizará $\frac{n(n-1)}{2}$ comparações a fim de ordernar totalmente a lista. Uma observação pertinente é que a quantidade de comparações a ser feita independe da porcentagem de ordenação da lista.

A seguir, segue a implementação do código em Python.

\begin{python}
def selection(V, n):
    for i in range(0, n):
        smallest_num_index = i
        for j in range(i, n):
            if V[j] < V[smallest_num_index]:
                smallest_num_index = j
        V[i], V[smallest_num_index] = V[smallest_num_index], V[i]

\end{python}

Com o código, percebe-se que, mesmo a lista já estando ordenada, o algoritmo continuará realizando os comandos. Assim, ...