\section{Algoritmo de contagem}
Diferentemente dos algoritmos outrora apresentados, o algoritmo em tópico possui uma complexidade linear, isto é, o tempo levado para ordernar uma lista, ao contrário dos apresentados, não cresce de maneira quadrática. 

Em relação ao uso de memória, contudo, há uma grande desvantagem no algoritmo, uma vez que a lista auxiliar utilizada possuirá $\max{lista}-\min{lista} $ elementos; ou seja, caso o maior elemento seja 5000 e o menor 1000, a lista auxiliar terá um tamanho de 4000 elementos.

Dessa forma, o algoritmo possui um melhor proveito se utilizado para listas com pouca variação de tamanho entre os seus elementos.

Segue, a seguir, os valores de \code{V} a cada modificação:
\begin{lstlisting}
[4, 3, 2, 1, 0]
[0, 3, 2, 1, 0]
[0, 1, 2, 1, 0]
[0, 1, 2, 1, 0]
[0, 1, 2, 3, 0]
[0, 1, 2, 3, 4]
\end{lstlisting}

Notoriamente, a lista, durante as mudanças intermediárias, não possui todos os valores iniciais, isto é, alguns dos elementos (como o 4) deixam de existir em alguma etapa e voltam a aparecer somente depois. Esta conduta é explicada pela diferença em que o código é executado: 
%TODO dizer que é criado um histograma e tudo mais.