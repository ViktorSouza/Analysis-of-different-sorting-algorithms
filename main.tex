\documentclass[10pt,a4paper]{article}
\setlength{\parskip}{\baselineskip}%
% \setlength{\parindent}{0pt}%
\usepackage{indentfirst}
\usepackage{url}
% \usepackage{minted}
% \usemintedstyle{manni}
\usepackage{listings}

% \usepackage{siunitx}
% \sisetup{table-number-alignment=center, exponent-product=\cdot}




\usepackage{xcolor} % for setting colors
\usepackage[most]{tcolorbox}



\definecolor{light-gray}{gray}{0.95}
\newcommand{\code}[1]{\colorbox{light-gray}{\lstinline{#1}}}


\lstset{%
  language=python,
  commentstyle=\bfseries,
  escapeinside={(*@}{@*)}
}

\definecolor{codegreen}{rgb}{0,0.6,0}
\definecolor{codegray}{rgb}{0.5,0.5,0.5}
\definecolor{codecyan}{rgb}{0.2, 0.4, 0.7}
\definecolor{backcolour}{rgb}{0.96,0.96,0.98}
\definecolor{dark}{rgb}{0.3,0.3,0.3}


\lstdefinestyle{mystyle}{
    backgroundcolor=\color{backcolour},   
    rulecolor=\color{backcolour},
    commentstyle=\color{codegreen},
    framexleftmargin=1pt,
    numbers=left,
    xleftmargin=2em,
    frame=single,
    framexleftmargin=1.5em,
     framextopmargin=1mm,
     framexbottommargin=1mm,
    keywordstyle=\color{codecyan},
    identifierstyle = \color[rgb]{0.0,0.4,0.4},
    numberstyle=\color{codegray}\ttfamily,
    stringstyle=\color{codecyan},
    basicstyle=\ttfamily\color{dark},
    breakatwhitespace=false,         
    breaklines=true,                 
    captionpos=b,                    
    keepspaces=true,           
    numbersep=5pt,                  
    showspaces=false,                
    showstringspaces=false,
    showtabs=false,                  
    tabsize=2,
}
\lstset{style=mystyle}
% \usepackage{pythonhighlight}
\usepackage[utf8]{inputenc}
\usepackage[portuguese]{babel}
\usepackage[T1]{fontenc}
\usepackage{amsmath}
\usepackage{graphicx}

\usepackage{amsfonts}
\usepackage{amssymb}
\graphicspath{ {./plots/} }
\author{João Viktor Souza Almeida}
\title{Análise de diferentes algoritmos de ordenação}



\begin{document}
\begin{titlepage} %iniciando a "capa"
    \begin{center} %centralizar o texto abaixo
    {\large Universidade de São Paulo}\\[0.2cm] %0,2cm é a distância entre o texto dessa linha e o texto da próxima
    {\large Instituto de Matemática e Estatística}\\[0.2cm] % o comando \\ "manda" o texto ir para próxima linha
    {\large Departamento de Ciência da Computação}\\[0.2cm]
    {\large Bacharelado em Ciência da Computação}\\[0.2cm]
    {\large MAC110 - Introdução à Computação}\\[5.1cm]
    {\bf \huge Análise de diferentes algoritmos de ordenação}\\[5.1cm] 
    \end{center} %término do comando centralizar
    {\large Aluno: João Viktor Souza Almeida}\\[0.7cm] % o comando \large deixa o texto grande
    {\large NUSP: 15521614}\\[0.7cm] % o comando \large deixa o texto grande
    {\large Turma: MAC0110-145-2024}\\[0.7cm] % o comando \large deixa o texto grande
    {\large Professor: Roberto Hirata Junior}\\[5.1cm]
    \end{titlepage} %término da "capa"

\subsection*{Resumo}
O relatório a seguir visa dissertar acerca algoritmos de ordenação clássicos, tais como bolha, contagem, inserção e seleção, bem como investigar exemplos, apresentar a implementação em Python e elucidar como o algoritmo se comporta ao ordenar uma lista.
Além disso, é verificado como a eficiência dos algoritmos variam dadas diferentes condições iniciais, tais como porcentagem de ordenação e composição da lista a ser ordenada. 
Por fim, foram implementados os algoritmos na linguagem C a fim de analisar os seus comportamentos em uma linguagem compilada, em contraste com a linguagem utilizada, Python, que é interpretada.
\

\subsection*{Abstract}
The following report seeks to dissertate the classic ordering algorithms, such as bubble, counting, insertion, and selection, as well as investigate examples, present Python's implementation, and elucidate how each algorithm comports when ordering a list.
In addition, it is verified how the efficiency varies given different initial conditions, like the ordination percentage and the list composition.
Finally, the algorithms were implemented in C to analyze how they comport in a compiled language, contrasting with Python, an interpreted language.

\section*{Metodologia}
Na criação do relatório, os testes dos algoritmos foram realizados utilizando a linguagem de programação Python, com a versão 3.10.11, no sistema operacional Windows.
Além disso, os arquivos em C foram compilados utilizando o GCC na versão 13.1.0.

Durante a execução dos algoritmos, foi utilizada uma máquina com as seguintes configurações:

\begin{itemize}
\item    Processador (CPU): Ryzen 5 3350G 
\item Memória (RAM): 16GB DDR4 @ 3200MHz
\item Armazenamento: 256GB SSD
\item Placa de Vídeo (GPU): Radeon Vega 11
\item Sistema Operacional: Windows 11 Pro
\end{itemize}
% TODO revisar o código
Adicionalmente, a fim de minimizar possíveis interferências nos resultados dos testes, os códigos foram executados com o mínimo de programas em segundo plano. 
Para computar a média de execução, foi criada uma função na qual que retornava a média e o desvio padrão das 10 itinerações do mesmo algoritmos.
Além disso, foram ignoradas possíveis margens de erros da função que embaralha as listas. Contudo, essa margem de erro torna-se desprezível devido ao tamanho das listas utilizadas.

Por fim, na implementação dos códigos na linguagem C, foi importada a biblioteca externa \code{stdbool.h} a fim de criar variáveis booleanas. Além disso, para possibilitar a importação das funções para o arquivo em Python, foi executado, no terminal, o comando \code{gcc -shared -fPIC -o libsortings.so main.c}, cuja função foi transformar as funções em uma biblioteca, possibilitando a invocação destes no Python.

\noindent\textbf{Palavras-chave:} insertion, bubble, counting, selection, algoritmos, ordenação, análise;

\subsection*{Testes}
    Para a realização do relatório, foram realizados dois testes. 
    O primeiro focou em observar os impactos causados pelo tamanho da lista a ser ordenada, ou seja, como o desvio padrão e a média mudam ao aumentar ou diminuir o tamanho da lista a ser ordenada por cada algoritmo. 
    Para isso, foi criada uma malha de repetição e computados as médias e os desvios-padrões de cada algoritmo com listas de tamanhos 1000, 5000, 10000, 50000 e 100000.
    
    O segundo, por outro lado, visou elucidar acerca das mudanças causadas pela taxa de ordenação de uma lista.
    Para isso, foi estabelecida uma segunda malha de repetição, na qual também foram computados as médias e os desvios-padrões de cada algoritmo, mas com uma lista de 100000 elementos e com porcentagens de ordenação de 1\%, 3\%, 5\%, 10\% e 50\%.
    

    Para computar a média, foi criada uma função cujos parâmetros são uma lista e o tamanho desta, respectivamente. O desvio padrão, sob o mesmo ponto de vista, foi calculado por meio de uma função que recebe os mesmos parâmetros que a função outrora citada.
    


\newpage
\tableofcontents
\section{Algoritmo de seleção}
O algoritmo de Seleção é um algoritmo de ordenação no qual, a cada itineração, o menor elemento da lista é garantido estar na posição correta. Ou seja, na primeira itineração, assegura-se que o menor elemento ficará na primeira posição da lista, na segunda itineração, o segundo elemento, e assim por diante. 
\vspace{\baselineskip}

Acerca do seu desempenho, o algoritmo em questão possui um "desempenho" quadrático, isto é, para $n$ elementops da lista, o algoritmo realizará $\frac{n(n-1)}{2}$ comparações a fim de ordernar totalmente a lista. Uma observação pertinente é que a quantidade de comparações a ser feita independe da porcentagem de ordenação da lista.

A seguir, segue a implementação do código em Python.

\begin{python}
def selection(V, n):
    for i in range(0, n):
        smallest_num_index = i
        for j in range(i, n):
            if V[j] < V[smallest_num_index]:
                smallest_num_index = j
        V[i], V[smallest_num_index] = V[smallest_num_index], V[i]

\end{python}

Com o código, percebe-se que, mesmo a lista já estando ordenada, o algoritmo continuará realizando os comandos, ou seja, o número de passagens a ser feita não é mudado devido às condições iniciais de ordenação da lista.
\subsection*{Quantidade de comparações}
Nesta subseção, irá ser debatida a quantidade de comparações feitas pelo algoritmo em questão. Como a quantidade de comparações é "constante" neste algoritmo, será "fácil" de analisá-la; nos outros algoritmos, contudo, a análise é mais compléxa devido à oscilação na quantidade de comparações.

Suponha-se que o algoritmo recebeu uma lista com $n$ elementos. Assim, como não há nenhuma restrição nas malhas de repetição a não ser ..., tem-se que, a primeira malha de repetição será executada $n$ vezes e, por consequência, em cada uma dessas execuções, a segunda malha será executada $n-i$ vezes (isto é, na primeira execução, o código contido na segunda malha de repetição será executado $n$ vezes, na segunda, $n-1$, e assim por diante.), O código será executado $n+(n-1)+(n-2)+...+(1)$, ou seja, $\sum_{i=1}^n i = \frac{n(n-1)}{2}$
%TODO ver depois se está correto
\section{Algoritmo de bolha}
O algoritmo de bolha é um algoritmo cuja complexidade é, assim como o anterior, quadrática. Sua principal característica é que, na n-ésima passagem pela lista, o algoritmo assegura estarem ordenados os n últimos elementos da lista.
Um fato interessante é que o nome do algoritmo faz referência às bolhas em bebidas, uma vez que as bolhas maiores sobem mais rapidamente do que as menores.


\subsection{Exemplo}
Seja $M$ uma lista [4,3,2,1,0].
Na primeira passagem da primeira malha de repetição, será detectado que os elementos 4 e 3 estão em posições erradas e, assim serão trocados, deixando a lista como [3,4,2,1,0]. Novamente, será detectado que os elementos 4 e 2 estão trocados e, assim cmomo no primeio passo, 

O algoritmo continuará realizando esses passos, isto é, trocar o elemento $j$ com o $j+1$ se o primeiro for maior que o segundo, até que, na 5º passagem, (pois a lista possui 5 elementos), a lista estará totalmente ordenada.

Destarte, seguem as impressões da lista a cada modificação realizada. 

\begin{lstlisting}
[4, 3, 2, 1, 0]
[3, 4, 2, 1, 0]
[3, 2, 4, 1, 0]
[3, 2, 1, 4, 0]
[3, 2, 1, 0, 4]
[2, 3, 1, 0, 4]
[2, 1, 3, 0, 4]
[2, 1, 0, 3, 4]
[1, 2, 0, 3, 4]
[1, 0, 2, 3, 4]
[0, 1, 2, 3, 4]
\end{lstlisting}
%TODO adicionar, de fato, todas as mudanças

Perceptivelmente, o maior elemento é, a cada modificação, levado uma posição para a direita até estar posição correta. 
\newpage
\subsection{Implementação}
Segue, a seguir, o código utilizado na implementação do algoritmo em questão em Python:

\begin{lstlisting}
def bubble(V, n):
    lim = n - 1
    while lim >= 0:
        isIncreasing = True
        for j in range(lim):
            if V[j] > V[j + 1]:
                isIncreasing = False
                V[j], V[j + 1] = V[j + 1], V[j]
        if(isIncreasing==True): 
                break
        lim -= 1
\end{lstlisting}


Caso a lista já esteja ordenada, o \code{isIncreasing} continuará, na primeira execução da malha de repetição, com o valor \code{True} e, por isso, o comando \code{break} será executado, parando, assim, a ordenação.

Constata-se, dessa forma, que, ao contrário do algoritmo anterior, este não realizará todas as etapas caso a lista já esteja ordenada, pois a variável \code{isIncreasing} funciona como um verificador.

\subsection{Quantidade de comparações}
Como a quantidade de comparações realizadas pelo bolha depende da lista, analisar-se-ão os casos. 

Caso a lista com $n$ elementos já esteja 100\% ordenada, o algoritmo detectará a ordenação na primeira passagem, pois a variável \code{isIncrease} continuará com o seu valor booleano verdadeiro após a malha de repetição mostrada na implementação, ou seja, serão realizadas $n$ comparações.

Por outro lado, na hipótese da lista está totalmente desordenada, haverá $n(n+1)\over 2$ comparações, pois, pelo mesmo motivo do inserção, serão feitas $n$ comparações na primeira itineração, $n-1$ na segunda, e assim sucessivalemente até a última, na qual será feita 1 comparação.

Finalmente, nos outros casos, a análise é mais complexa e depende de outros fatores que não estão no escopo do relatório. Contudo, em casos nos quais a lista não está nem ordenada nem totalmente desordenada, o algoritmo possui um comportamento quadrático\cite{bubblecomplexity}.
\section{Algoritmo de inserção}
O algoritmo de inserção, assim como o de bolha, varia com a porcentagem de ordenação da lista recebida, ou seja, caso a lista já esteja ordenada, não haverá nada a ser feito. 


O algoritmo em questão funcionam da seguinte forma: a lista começa dada como ordenada até que se ache um $j$ tal que $V[j]>V[j+1]$, onde $j$ é um inteiro maior que zero e menor do que o tamanho da lista menos um. 
Caso isso ocorra, o algoritmo trocará os dois valores e comparará, da mesma forma, $V[j-1]$ e $V[j]$. Quando o valor da antiga posição $j$ for menor do que a posição sucessora, garante-se, então, que a lista está ordenada de 0 até $j+1$(ver isto depois). Contudo, caso não exista um $j$, então o algoritmo indica que a lista já está ordenada e, assim, evita mais comparações.
\\

Logo, tomando uma lista $[4,3,2,1,0]$, caso esta fosse impressa a cada execução do algoritmo, os resultados seriam:
Destarte, caso fosse impresso a lista a cada modificação, os resultados seriam os seguintes:
\begin{lstlisting}
[4, 3, 2, 1, 0]
[3, 4, 2, 1, 0]
[3, 2, 4, 1, 0]
[2, 3, 4, 1, 0]
[2, 3, 1, 4, 0]
[2, 1, 3, 4, 0]
[1, 2, 3, 4, 0]
[1, 2, 3, 0, 4]
[1, 2, 0, 3, 4]
[1, 0, 2, 3, 4]
\end{lstlisting}

Como pode-se perceber, quando é encontrado um elemento $j$ menor do que o anterior, este é "levado" para trás até que esteja na posição correta

\begin{lstlisting}
def insertion(V, n):
last_index = 0
for i in range(last_index,n - 1):
    if V[i] > V[i + 1]:
        j = i
        while V[j] > V[j + 1]:
            V[j + 1], V[j] = V[j], V[j + 1]
            if j <= 0:
                break
            j -= 1
\end{lstlisting}
\section{Algoritmo de contagem}
Diferentemente dos algoritmos outrora apresentados, o algoritmo em tópico possui uma complexidade linear, isto é, o tempo levado para ordernar uma lista, ao contrário dos apresentados, não cresce de maneira quadrática. 

Em relação ao uso de memória, contudo, há uma grande desvantagem no algoritmo, uma vez que a lista auxiliar utilizada possuirá $\max{lista}-\min{lista} $ elementos; ou seja, caso o maior elemento seja 5000 e o menor 1000, a lista auxiliar possuirá de 4000 elementos.

Dessa forma, o algoritmo possui um melhor proveito se utilizado para listas com pouca variação de tamanho entre os seus elementos.

\subsection{Exemplo}
Seja $M$ a lista \[4,2,3,1,4,2,2,0\] Como a diferença entre o maior e o menor é elemento é 4, a lista auxiliar $Aux$ será de tamanho 4.
Assim, a lista auxiliar será da seguinte forma (assumindo que o primeiro elemento é $Aux[0]$): o primeiro elemento terá o valor igual à quantidade de zeros na lista original, que é 1, o segundo elemento de $Aux$, por outro lado, terá o mesmo valor 3, uma vez que há três elemento 2 na lista original. Essa lógica seguirá até chegar no último elemento.
Assim, de início, a lista $Aux$ será [1,1,3,1,2].

Após isso, começará a modificação da lista original: será adicionados a $n$ vezes o elemento referente ao índice da lista auxiliar, ou seja, como $Aux[0]$ é igual a 1, será adicionado um zero á lista original (adicionando pela direita um ao lado do outro). Em $Aux[2]$, por exemplo, o elemento é 3, assim, será adicionado o elemento 2 três vezes seguidas à lista $M$.

Segue, a seguir, os valores de \code{Aux} a cada modificação e, logo abaixo, os valores de $M$:
\begin{lstlisting}
[0, 0, 0, 0, 0]
[1, 0, 0, 0, 0]
[1, 1, 0, 0, 0]
[1, 1, 3, 0, 0]
[1, 1, 3, 1, 0]
[1, 1, 3, 1, 2]
\end{lstlisting}

\begin{lstlisting}
    [4, 2, 3, 1, 4, 2, 2, 0]
    [0, 2, 3, 1, 4, 2, 2, 0]
    [0, 1, 3, 1, 4, 2, 2, 0]
    [0, 1, 2, 1, 4, 2, 2, 0]
    [0, 1, 2, 2, 4, 2, 2, 0]
    [0, 1, 2, 2, 2, 2, 2, 0]
    [0, 1, 2, 2, 2, 3, 2, 0]
    [0, 1, 2, 2, 2, 3, 4, 0]
    [0, 1, 2, 2, 2, 3, 4, 4] 
\end{lstlisting}

Notoriamente, a lista, durante as mudanças intermediárias, não possui todos os valores iniciais, isto é, alguns dos elementos (como o 4) deixam de existir em alguma etapa e voltam a aparecer somente depois.


\subsection{Implementação}
Segue a implementação em Python

\begin{lstlisting}
def counting(V, n):
    max_element = max(V)
    hist_list = [0 for _ in range(max_element + 1)]
    for i in range(max_element + 1):
        hist_list[i] = count_element_in_array(i, V)
    index = 0
    for i in range(max_element + 1):
        for _ in range(hist_list[i]):
            V[index] = i
            index += 1
\end{lstlisting}
Como pode-se perceber, o algoritmo de contagem, assim como os outros, possui duas malhas de repetição. Contudo, em vez da segunda estar dentro da terceira, uma ocorre após a outra.
Apesar disso, uma singularidade desse algoritmo é que, ao contrário dos outros, a quantidade de itinerações na primeira malha de repetição depende do tamanho do maior elemento (pensando apenas na implementação com números positivos), tornando-o eficaz para listas grandes mas com elementos menores.
Adicionalmente, este algoritmo não ordena os elementos por comparação...

\subsection{Quantidade de comparações}
Diferentemente de todos os algoritmos outrora citados, este não realiza qualquer comparação.


\section{Resultados}
Nesta seção, será debatido acerca dos resultados obtidos por meio dos dois testes realizados. O primeiro visou analisar o tempo médio de cada algoritmo, bem como a sua variância. 
No segundo teste, por outro lado, buscou-se avaliar o comportamento dos algoritmos bolha e inserção quando submetidos a vetores com diferentes taxas de ordenação, sendo estas 1\%, 3\%, 5\%, 10\% e, por fim, 50\%. 


\subsection{Variação na quantidade de elementos}
Como comentado, os resultados comprovam o comportamento quadrático do bolha, seleção e inserção. 
Entretanto, é preciso ressaltar que, se existisse uma variação significativa nos números da lista a ser ordenada, o algoritmo de contagem apresentaria um tempo para ordenar maior, podendo ultrapassar o bolha, por exemplo.

\begin{figure}[h]
    \includegraphics[width=8cm]{sizes.png}
    \caption{Gráfico elucidando o comportamento de cada algoritmo em tópico. No eixo x, há a quantidade de elementos; no eixo y, o tempo médio gasto em segundos}
    \end{figure}

    \begin{table}[h]
        % \centering
        \begin{tabular}{llllll}
            \textbf{Tamanho da lista} & \textbf{Seleção} & \textbf{Inserção} & \textbf{Bolha} & \textbf{Contagem} \\
            1000 & 0.04597 & 0.05796 & 0.06942 & 0.24648 \\
            5000 & 0.98705 & 1.42168 & 1.84406 & 1.21472 \\
            10000 & 3.88312 & 5.79583 & 7.49110 & 2.90298 \\
            50000 & 97.74410 & 148.20082 & 194.80364 & 14.39026 \\
            100000 & 389.11648 & 596.82038 & 791.78906 & 28.75249 \\
        \end{tabular}
        \caption{Tempos de execução dos algoritmos de ordenação (em segundos) para diferentes tamanhos de lista}
        \label{tab:tempos_algoritmos}
    \end{table}

O bolha, apesar de variar dependendo da taxa de ordenação, possui o maior tempo necessário para ordenar. Em seguida, o algoritmo de inserção detém o segundo maior período para organizar a sequência, e isso se dá pelo fato de que o inserção possui uma complexidade quadrática independente da posição inicial dos elementos. 

O algoritmo de seleção, como esperado, contém o melhor tempo de resposta dentre os algoritmos de comparação, tendo em vista que, além de ser mais eficiente em listas com alguns elementos ordenados, realiza menos permutações que o bolha. 

Por último, o contagem apresentou o melhor desempenho dos implementados. Contudo, é necessário destacar que o contagem obteve este performance devido à faixa de números aleatórios escolhida (de 0 a 9999), pois, como analisado, o algoritmo em questão varia com os números recebidos.

Assim, percebe-se que, para listas com muitos elementos, o algoritmo contagem é o mais eficiente entre os implementados, enquanto o bolha apresentou o resultado menos satisfatório para o teste.


\subsection{Ordenação prévia}
\begin{figure}[h]
    \includegraphics[width=8cm]{percentages.png}
    \caption{Gráfico ilustrando o tempo médio 'gasto' dos algoritmos bolha e inserção em relação à taxa de ordenação inicial dos vetores.}
\end{figure}
\begin{table}[h]
    \begin{tabular}{lcc}
        \textbf{Tamanho da lista} & \textbf{Bolha} & \textbf{Contagem} \\
        1000 & 3.4037e-08 & 1.0778e-05 \\
        5000 & 1.3713e-04 & 5.0052e-05 \\
        10000 & 1.5767e-03 & 1.3368e-03 \\
        50000 & 4.9661 & 0.0697 \\
        100000 & 15.7575 & 1.2251 \\
    \end{tabular}
    \caption{Desvios padrão dos tempos de execução dos algoritmos de ordenação (em segundos) para diferentes tamanhos de lista}
    \label{tab:desvios_algoritmos}
\end{table}

Inicialmente, quando ambos os algoritmos são submetidos a uma lista já ordenada, o tempo de execução do algoritmo de seleção é quase imediato; o bolha, por outro lado, possui uma duração considerável para verificar que a lista já está ordenada. Adicionalmente, ao aumentar a permutação dos elementos, o bolha mantém a sua alta duração, principalmente devido à quantidade de comutações realizadas por este.

Por outro lado, o algoritmo de seleção apresentou um crescimento maior quando intensificada a desordenação da lista, uma vez que a quantidade de comutações realizadas pelo algoritmo tende a se aproximar a do bubble, ou seja, quanto mais ordenada a lista, mais eficiente o seleção é em comparação ao segundo.

Portanto, o algoritmo de seleção é mais eficiente do que o bolha, ainda quando submetido a sequências com diferentes taxas de ordenação. Contudo, o seleção possui uma taxa de crescimento ligeiramente maior do que o bolha quando aumentada a porcentagem de embaralhamento.

\section{Implementação em C}
Nesta seção, buscar-se-á analisar os algoritmos comentados em C a fim de analisar os impactos causados no desempenho quando executado um algoritmo em uma linguagem interpretada, tal como o Python.

Após visto o comportamento dos algoritmos quando implementados em Python, uma linguagem interpretada, percebe-se que, mesmo para vetores pequenos, o tempo gasto para ordená-los é significativo.

Quando implementados utilizando a linguagem C, contudo, percebe-se uma considerável redução na duração dos testes, uma vez que, em razão do C ser uma linguagem compilada, as malhas de repetições tendem a ser mais eficientes.
% TODO por que são mais eficientes?
Ademais, ao utilizar a linguagem C, há um maior controle na forma como o código será executado, devido ao C ser uma linguagem com um nível mais baixo.
% TODO ou seja, ...
Dessa forma, serão analisados os desempenhos de alguns dos algoritmos já implementados em Python, mas, desta vez, estes serão escritos na linguagem C.

A fim de realizar o testes, os algoritmos foram implementados na linguagem C, utilizando, como referência, os códigos já escritos em Python. Sendo assim, tendo em vista que os códigos utilizandos em C não foram otimizados e nem modificados de forma a deixá-los mais lentos, ter-se-á uma comparação equânime.

Além disso, as funções foram invocadas no Python visando criar os gráficos dos resultados.

Foram realizados somentes testes visando analisar o comportamento dos algoritmos em C quando estes são submetidos a listas com tamanhos variáveis, ou seja, o segundo teste não será executado pois..
\newpage
\subsection{Resultados}
\begin{figure}[h]
    \includegraphics[width=8cm]{c sizes.png}
    \caption{Gráfico ilustrando o tempo médio de execução dos algoritmos implementados em C comparado ao do contagem, implementado em Python.}
\end{figure}
Perceptivelmente, o ganho de desempenho dos algoritmos implementados em C mostrou-se singificativo, mesmo os mais lentos quando implementados em Python, como o bolha, obtiveram uma performance superior ao contagem da linguagem inicial.

De fato, o contagem implementado em C mostrou-se *\% mais eficiente do que a sua versão em Python. Dessa maneira, ....

Ademais, nota-se que, ainda que com listas pequenas, o desempenho dos algoritmos em C em relação aos do Python permaneceu quase constante, isto é, ao aumentar o tamanho das listas, a implementação em C não obteve um ganho nem uma perda expressiva. Consequentemente, ...

Portanto, visando um ganho de performance, é recomendável a utilização da linguagem C a fim de implementar algoritmos que se "aproveitam" de malhas de repetições.
Caso seja necessário o uso do Python para outras tarefas no projeto, as funções em C podem ser chamadas por meio de ...

\section{Conclusão}
Desta maneira, apesar de deterem diferentes desempenhos dependendo da ordenação do vetor, os algoritmos em questão possuem um baixo proveito com listas muito grandes. Portanto, a fim de ordenar uma lista com maior rapidez, é aconselhável utilizar a função já implementada do Python, conhecida como timsort.

Adicionalmente, o algoritmo de contagem, ainda que o vetor passado tenha poucos elementos, pode ter um tempo de execução muito alto em razão da variação entre os números. Assim, é aconselhável a utilização do algoritmo em pauta apenas com listas com uma exígua variação.

Em outra perspectiva, o bolha foi o algoritmo que apresentou o menor rendimento quando submetido a sequências com diferentes tamanhos; o contagem, por outro lado, garantiu, apesar da faixa de valores maior, o melhor desempenho, principalmente nas listas com mais elementos. Assim, a utilização do bolha é recomendada apenas para fins didáticos.


Por fim, diante das comparações realizadas entre o Python e C, observa-se que, para algoritmos que dependem de malhas de repetição, a segunda linguagem apresenta um desempenho mais satisfatório.

\section{Considerações finais}
A realização deste relatório possibilitou o aprendizado em diversas áreas do conhecimento, especialmente na análise de algoritmos, ao separá-los por casos e verificá-los individualmente.
Além disso, o projeto proporcionou uma ampla experiência na criação de textos acadêmicos e na investigação da eficiência e do comportamento de um algoritmo.
Portanto, o exercício programa permitiu um aprimoramento em competências que serão de extrema importância para futuros projetos.


\section{Considerações finais}
A realização deste relatório possibilitou o aprendizado em diversas áreas do conhecimento, especialmente na análise de algoritmos, ao separá-los por casos e verificá-los individualmente.
Além disso, o projeto proporcionou uma ampla experiência na criação de textos acadêmicos e na investigação da eficiência e do comportamento de um algoritmo.
Portanto, o exercício programa permitiu um aprimoramento em competências que serão de extrema importância para futuros projetos.

\newpage
\begin{thebibliography}{3}
    \bibitem{bubblecomplexity}
    Bubble Sort Time Complexity and Algorithm Explained, builtin, 2023. Disponível em: \url{https://builtin.com/data-science/bubble-sort-time-complexity#:~:text=The%20bubble%20sort%20algorithm%27s%20average,complexity%3A%20O(n%C2%B2)}. Acesso em: 08 de jun. de 2024.
    \bibitem{insertioncomplexity}
    Insertion Sort Explained–A Data Scientists Algorithm Guide, 2021. Disponível em: \url{https://developer.nvidia.com/blog/insertion-sort-explained-a-data-scientists-algorithm-guide/#:~:text=The%20worst%2Dcase%20(and%20average,O(n)%20time%20complexity.}. Acesso em: 08 de jun. de 2024.
\end{thebibliography}


\end{document}