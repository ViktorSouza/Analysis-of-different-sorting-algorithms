\documentclass[10pt,a4paper]{article}
\usepackage{indentfirst}
\usepackage[utf8]{inputenc}
\usepackage[portuguese]{babel}
\usepackage[T1]{fontenc}
\usepackage{amsmath}
\usepackage{amsfonts}
\usepackage{amssymb}
\author{João Viktor Souza Almeida}
\title{Análise de diferentes algoritmos de ordenação \[x^2-3+5\]}



\begin{document}
\maketitle
\
\textbf{Palavras-chave:} insertion, bubble, counting, selection, algoritmos, ordenação, análise;

\tableofcontents
\section{Algoritmo de Seleção}
O algoritmo de Seleção é um algoritmo de ordenação no qual, a cada itineração, o menor elemento da lista é garantido estar na posição correta. Ou seja, na primeira itineração, o menor elemento ficará na primeira posição da lista, na segunda itineração, o segundo elemento, e assim por diante. 
\par
Acerca do seu desempenho, o algoritmo em questão possui um "desempenho" quadrático, isto é, para $n$ números da lista, o algoritmo fará $n(n-1)$ comparações a fim de ordernar totalmente a lista. Uma observação pertinente é a quantidade de comparações a ser feita independe da porcentagem de ordenação da lista.
% descrever o porquê isto ocorre.


\end{document}