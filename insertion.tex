\section{Algoritmo de inserção}
O algoritmo de inserção, assim como o de bolha, varia com a porcentagem de ordenação da lista recebida, ou seja, caso a lista já esteja ordenada, não haverá nada a ser feito. 


O algoritmo em questão funcionam da seguinte forma: a lista começa dada como ordenada até que se ache um $j$ tal que $V[j]>V[j+1]$, onde $j$ é um inteiro maior que zero e menor do que o tamanho da lista menos um. 
Caso isso ocorra, o algoritmo trocará os dois valores e comparará, da mesma forma, $V[j-1]$ e $V[j]$. Quando o valor da antiga posição $j$ for menor do que a posição sucessora, garante-se, então, que a lista está ordenada de 0 até $j+1$(ver isto depois). Contudo, caso não exista um $j$, então o algoritmo indica que a lista já está ordenada e, assim, evita mais comparações.
\\

Logo, tomando uma lista $[4,3,2,1,0]$, caso esta fosse impressa a cada execução do algoritmo, os resultados seriam:
Destarte, caso fosse impresso a lista a cada modificação, os resultados seriam os seguintes:
\begin{lstlisting}
[4, 3, 2, 1, 0]
[3, 4, 2, 1, 0]
[3, 2, 4, 1, 0]
[2, 3, 4, 1, 0]
[2, 3, 1, 4, 0]
[2, 1, 3, 4, 0]
[1, 2, 3, 4, 0]
[1, 2, 3, 0, 4]
[1, 2, 0, 3, 4]
[1, 0, 2, 3, 4]
\end{lstlisting}

Como pode-se perceber, quando é encontrado um elemento $j$ menor do que o anterior, este é "levado" para trás até que esteja na posição correta

\begin{lstlisting}
def insertion(V, n):
last_index = 0
for i in range(last_index,n - 1):
    if V[i] > V[i + 1]:
        j = i
        while V[j] > V[j + 1]:
            V[j + 1], V[j] = V[j], V[j + 1]
            if j <= 0:
                break
            j -= 1
\end{lstlisting}