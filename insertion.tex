\section{Algoritmo de inserção}
O algoritmo de inserção possui, como principal característica, a asseguração de que, a cada itineração $n$, os n-ésimos primeiros elementos estarão ordenados. Além disso, assim como o de bolha, este varia com a porcentagem de ordenação da lista recebida, ou seja, caso a lista já esteja ordenada, será detectado logo na primeira passagem e, assim, não haverá nada a ser feito. 


\subsection{Exemplo}
Seja, para fins de exemplificação, $V$ a lista $[4,3,2,1,0]$.  

O algoritmo em questão funcionará da seguinte forma: a sequência começa dada como ordenada até que se ache um $j$ tal que $V[j]>V[j+1]$, onde $j$ é um inteiro maior ou igual a zero e menor que o tamanho da lista menos um. 
Caso isso ocorra, o algoritmo trocará os dois valores e comparará, da mesma forma, $V[j-1]$ e $V[j]$. Quando o valor da antiga posição $j$ for menor do que a posição sucessora, garante-se, então, que a sequência está ordenada de 0 até $j+1$. Contudo, caso não exista um $j$, então o algoritmo indica que a lista já está ordenada e, assim, evita mais comparações.
\\

Logo, tomando o vetor $M$, caso este fosse impresso a cada modificação, os resultados seriam:
\begin{lstlisting}
[4, 3, 2, 1, 0]
[3, 4, 2, 1, 0]
[3, 2, 4, 1, 0]
[2, 3, 4, 1, 0]
[2, 3, 1, 4, 0]
[2, 1, 3, 4, 0]
[1, 2, 3, 4, 0]
[1, 2, 3, 0, 4]
[1, 2, 0, 3, 4]
[1, 0, 2, 3, 4]
[0, 1, 2, 3, 4]
\end{lstlisting}

Visivelmente, observa-se que, sempre que é encontrado um elemento $e$ menor que o anterior, ambos são trocados e, em seguida, o algoritmo começa a verificar e mover $e$ para trás até o elemento em questão fique em sua posição correta.
\newpage

\subsection{Implementação}
Como se pode perceber, quando é encontrado um elemento $j$ menor do que o anterior, este é levado para trás até que esteja na posição correta

\begin{lstlisting}
def insertion(V, n):
    last_index = 0
    for i in range(last_index,n - 1):
        if V[i] > V[i + 1]:
            j = i
            while V[j] > V[j + 1]:
                V[j + 1], V[j] = V[j], V[j + 1]
                if j <= 0:
                    break
                j -= 1
\end{lstlisting}

\subsection{Quantidade de comparações}
Tendo em vista que as comparações realizadas dependem do vetor, os casos serão analisados. 

Caso a lista com $n$ elementos esteja em ordem, o algoritmo detectará na primeira itineração devido à permanência do valor verdadeiro da variável \code{isIncreasing}, ou seja, serão realizadas $n$ comparações.

Por outro lado, na hipótese da lista está totalmente desordenada, haverá $n(n+1)\over 2$. A prova é análoga à do bolha, uma vez que haverá $n$ comparações na primeira execução, $n-1$ na segunda, e assim por diante, ou seja, serão realizadas $n + (n-1)+(n-2)+\cdots+1 = {n(n+1)\over 2}$ comparações.

Finalmente, nos outros casos, a análise é, assim como no algoritmo anterior, complexa. Contudo, em cenários nos quais a sequência não se enquadra nos casos supracitados, o algoritmo possui um comportamento quadrático\cite{insertioncomplexity}.