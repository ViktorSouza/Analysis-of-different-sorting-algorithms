\section{Algoritmo de bolha}
O algoritmo de bolha é um algoritmo cuja complexidade é, assim como o anterior, quadrática. Sua principal característica é que, na n-ésima passagem pela lista, o algoritmo assegura estarem ordenados os n últimos elementos da lista.

\textbf{Exemplo}: Seja $M$ uma lista [4,3,2,1,0].
Na primeira passagem, o maior valor da lista (4) estará em seu lugar correto, ou seja, M será [3,2,1,0,4]. Na segunda passagem, o segundo maior elemento também estará posicionado em sua posição correta; assim, M será [2,1,0,3,4].

O algoritmo continuará realizando "isto" até que, na 5º passagem, (pois a lista possui 5 elementos), a lista estará totalmente ordenada.
Portanto, assim como no algoritmo anterior, o Bolha é eficiente apenas para listas com tamanhos pequenos.
