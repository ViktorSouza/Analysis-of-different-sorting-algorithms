\section{Algoritmo de bolha}
O algoritmo de bolha é um algoritmo cuja complexidade é, assim como o anterior, quadrática. Sua principal característica é que, na n-ésima passagem pela lista, o algoritmo assegura estarem ordenados os n últimos elementos da lista.

\textbf{Exemplo}: Seja $M$ uma lista [4,3,2,1,0].
Na primeira passagem, o maior valor da lista (4) estará em seu lugar correto, ou seja, M será [3,2,1,0,4]. Na segunda passagem, o segundo maior elemento também estará posicionado em sua posição correta; assim, M será [2,1,0,3,4].

O algoritmo continuará realizando "isto" até que, na 5º passagem, (pois a lista possui 5 elementos), a lista estará totalmente ordenada.

\begin{python}
def bubble(V, n):
    lim = n - 1
    while lim >= 0:
        isIncreasing = True
        for j in range(lim):
            if V[j] > V[j + 1]:
                isIncreasing = False
                V[j], V[j + 1] = V[j + 1], V[j]
        if(isIncreasing==True): 
                break
        lim -= 1
\end{python}

De acordo com o código, constata-se que, ao contrário do algoritmo anterior, este não realizará todas as etapas caso a lista já esteja ordenada, pois a variável $isIncreasing$ indica se a lista já se encontra ou não ordenada.