\section{Algoritmo de bolha}
O algoritmo de bolha é um algoritmo cuja complexidade é, assim como o anterior, quadrática. Sua principal característica é que, na n-ésima passagem pela lista, o algoritmo assegura estarem ordenados os n últimos elementos da lista.

\subsection{Exemplo}
Seja $M$ uma lista [4,3,2,1,0].
Na primeira passagem, o maior valor da lista (4) estará em seu lugar correto, ou seja, M será [3,2,1,0,4]. Na segunda passagem, o segundo maior elemento também estará posicionado em sua posição correta; assim, M será [2,1,0,3,4].
O algoritmo continuará realizando "isto" até que, na 5º passagem, (pois a lista possui 5 elementos), a lista estará totalmente ordenada.

Destarte, seguem as impressões da lista a cada modificação realizada. 

\begin{lstlisting}
[4, 3, 2, 1, 0]
[3, 4, 2, 1, 0]
[3, 2, 4, 1, 0]
[3, 2, 1, 4, 0]
[3, 2, 1, 0, 4]
[2, 3, 1, 0, 4]
[2, 1, 3, 0, 4]
[2, 1, 0, 3, 4]
[1, 2, 0, 3, 4]
[1, 0, 2, 3, 4]
\end{lstlisting}

Perceptivelmente, o maior elemento é, a cada modificação, levado uma posição para a direita, até que o elemento esteja na posição correta. De fato, o nome do algoritmo faz referência às bolhas em bebidas, uma vez que as bolhas maiores sobem mais rapidamente do que as menores.

\newpage
\subsection{Implementação}
Segue, a seguir, o código utilizado na implementação do algoritmo em questão em Python:

\begin{lstlisting}
def bubble(V, n):
    lim = n - 1
    while lim >= 0:
        isIncreasing = True
        for j in range(lim):
            if V[j] > V[j + 1]:
                isIncreasing = False
                V[j], V[j + 1] = V[j + 1], V[j]
        if(isIncreasing==True): 
                break
        lim -= 1
\end{lstlisting}


Perceptivelmente, caso a lista já esteja ordenada, na primeira execução da malha de repetição, o \code{isIncreasing} continuará com o valor \code{True} e, portanto, o comando \code{break} será executado, parando, assim, a ordenação.

Constata-se, dessa forma, que, ao contrário do algoritmo anterior, este não realizará todas as etapas caso a lista já esteja ordenada, pois a variável \code{isIncreasing} funciona como um "testador".

\subsection{Quantidade de comparações}
Como a quantidade de comparações realizadas pelo bolha depende da lista, analisar-se-ão os casos. 

Caso a lista com $n$ elementos já esteja 100\% ordenada, o algoritmo detectará "isto" na primeira passagem, pois a variável \code{isIncrease} continuará com o seu valor booleano verdadeiro após a malha de repetição mostrada na implementação, ou seja, serão realizadas $n$ comparações.

Por outro lado, na hipótese da lista está totalmente desordenada, haverá $n(n+1)\over 2$ comparações. 

Finalmente, nos outros casos, a análise é mais complexa e depende de outros fatores que não estão no escopo da análise. Contudo, em casos nos quais a lista não está nem ordenada nem totalmente desordenada, o algoritmo possui um comportamento quadrático\cite{bubblecomplexity}.
%TODO citar fonte