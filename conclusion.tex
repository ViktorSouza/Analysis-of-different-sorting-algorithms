\section{Conclusão}
Desta maneira, apesar de possuírem diferentes desempenhos dependendo da ordenação da lista, os algoritmos em questão, possuem um baixo proveito com listas muito grande. Portanto, a fim de ordenar uma lista com maior rapidez, é notável que a função built-in do Python deve, pelo menos para listas grandes, ser usada em relação aos outros métodos outrora discutidos.

Adicionalmente, o algoritmo de contagem, ainda que possua um comportamento linear em relação ao tamanho das listas, pode sofrer variações em razão da diferença entre o maior e o menor elemento da lista, isto é, ainda que a lista seja relativamente pequena, o tempo de realização do algoritmo pode ser muito alto caso dependendo da variação entre os números. Assim, é aconselhável a utilização do algoritmo em pauta apenas com listas que possuem uma exígua variação.

Em uma outra perspectiva, o bolha foi o algoritmo que apresentou o menor rendimento submetido a listas com diferentes tamanhos e taxas de ordenação. Dessa forma, o bolha é recomendado apenas para fins de demonstrações.