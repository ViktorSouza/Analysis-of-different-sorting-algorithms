\section{Algoritmo de contagem}
Diferentemente dos algoritmos outrora discutidos, o contagem possui uma complexidade linear, isto é, o tempo levado para ordenar cresce proporcionalmente com o tamanho da lista recebida. 

Em relação ao uso de memória, contudo, há uma grande desvantagem no algoritmo: devido ao uso de uma lista auxiliar, esta possuirá $\max({lista})-\min({lista}) $ elementos; ou seja, caso o maior elemento seja 5000 e o menor 1000, a lista auxiliar possuirá de 4000 elementos.

Dessa forma, o algoritmo possui um melhor proveito se utilizado para listas com pouca variação de tamanho entre os seus elementos.

\subsection{Exemplo}
Seja $M$ a lista $[4,2,3,1,4,2,2,0]$. Como a diferença entre o maior e o menor é elemento é 4, a lista auxiliar $Aux$ será de tamanho 4.
Assim, a lista auxiliar $Aux$ será da seguinte forma (assumindo que o primeiro elemento é $Aux[0]$): o primeiro elemento terá o valor igual à quantidade de zeros na lista original, que é 1, o segundo elemento também será 1 por motivo análogo, e o terceiro elemento de $Aux$, por outro lado, receberá o valor 3, uma vez que há três elementos 2 na lista original. Essa lógica prevalecerá até último elemento.
Assim, de início, a lista $Aux$ será [1,1,3,1,2].

Em seguida, começará a modificação da lista original: serão adicionados à lista original $n$ vezes o elemento referente ao índice da lista auxiliar, ou seja, como $Aux[0]$ é igual a 1, será adicionado um zero a $M$ (adicionando da esquerda para a direita e um ao lado do outro). Em $Aux[2]$, por exemplo, o elemento é 3 e, assim, será adicionado o elemento 2 três vezes seguidas à lista $M$.

Seguem os valores de \code{Aux} a cada modificação:
\begin{lstlisting}
[0, 0, 0, 0, 0]
[1, 0, 0, 0, 0]
[1, 1, 0, 0, 0]
[1, 1, 3, 0, 0]
[1, 1, 3, 1, 0]
[1, 1, 3, 1, 2]
\end{lstlisting}
\newpage
Para fins elucidativos, serão substituídos por "0" os elementos já existentes em $M$ a fim de mostrar os valores adicionados posteriormente, ou seja, supor-se-á que a lista $M$ teve todos os seus valores trocados para 0.
\begin{lstlisting}
[0, 0, 0, 0, 0, 0, 0, 0]
[0, 0, 0, 0, 0, 0, 0, 0]
[0, 1, 0, 0, 0, 0, 0, 0]
[0, 1, 2, 0, 0, 0, 0, 0]
[0, 1, 2, 2, 0, 0, 0, 0]
[0, 1, 2, 2, 2, 0, 0, 0]
[0, 1, 2, 2, 2, 3, 0, 0]
[0, 1, 2, 2, 2, 3, 4, 0]
[0, 1, 2, 2, 2, 3, 4, 4] 
\end{lstlisting}

Percebe-se, portanto, que a lista foi ordenada realizando-se nenhuma comparação.

\subsection{Implementação}
Segue a implementação do algoritmo em questão em Python:

\begin{lstlisting}
def counting(V, n):
    max_element = max(V)
    hist_list = [0 for _ in range(max_element + 1)]
    for i in range(max_element + 1):
        hist_list[i] = count_element_in_array(i, V)
    index = 0
    for i in range(max_element + 1):
        for _ in range(hist_list[i]):
            V[index] = i
            index += 1
\end{lstlisting}
Como se pode perceber, o algoritmo de contagem, assim como os outros, possui duas malhas de repetição. Contudo, em vez de as malhas estarem aninhada, uma ocorre após a outra.
Apesar disso, uma singularidade desse algoritmo é que, ao contrário dos citados, a quantidade de itinerações na primeira malha de repetição depende do tamanho do maior elemento (pensando apenas na implementação com números positivos), tornando-o eficaz para listas grandes e com elementos menores.

\subsection{Quantidade de comparações}
Diferentemente de todos os algoritmos outrora citados, este não realiza qualquer comparação.